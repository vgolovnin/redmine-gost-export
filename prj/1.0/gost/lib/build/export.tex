\begin{filecontents}{uspd.bib}
@book{gost101,
  title = {ГОСТ 19.101-77 Виды программ и программных документов},
  address = {М.},
  publisher = {ИПК Издательство стандартов},
  series = {Единая система программной документации},
  year = {2001}
}

@book{gost102,
  title = {ГОСТ 19.102-77 Стадии разработки},
  address = {М.},
  publisher = {ИПК Издательство стандартов},
  series = {Единая система программной документации},
  year = {2001}
}

@book{gost103,
  title = {ГОСТ 19.103-77 Обозначения программ и программных документов},
  address = {М.},
  publisher = {ИПК Издательство стандартов},
  series = {Единая система программной документации},
  year = {2001}
}

@book{gost104,
  title = {ГОСТ 19.104-78 Основные надписи},
  address = {М.},
  publisher = {ИПК Издательство стандартов},
  series = {Единая система программной документации},
  year = {2001}
}

@book{gost105,
  title = {ГОСТ 19.105-78 Общие требования к программным документам},
  address = {М.},
  publisher = {ИПК Издательство стандартов},
  series = {Единая система программной документации},
  year = {2001}
}

@book{gost106,
  title = {ГОСТ 19.106-78 Требования к программным документам, выполненным печатным способом},
  address = {М.},
  publisher = {ИПК Издательство стандартов},
  series = {Единая система программной документации},
  year = {2001}
}

@book{gost201,
  title = {ГОСТ 19.201-78 Техническое задание. Требования к содержанию и оформлению},
  address = {М.},
  publisher = {ИПК Издательство стандартов},
  series = {Единая система программной документации},
  year = {2001}
}

@book{gost301,
  title = {ГОСТ 19.301-79 Программа и методика испытаний. Требования к содержанию и оформлению},
  address = {М.},
  publisher = {ИПК Издательство стандартов},
  series = {Единая система программной документации},
  year = {2001}
}

@book{gost401,
  title = {ГОСТ 19.401-78 Текст программы. Требования к содержанию и оформлению},
  address = {М.},
  publisher = {ИПК Издательство стандартов},
  series = {Единая система программной документации},
  year = {2001}
}

@book{gost404,
  title = {ГОСТ 19.404-79 Пояснительная записка. Требования к содержанию и оформлению},
  address = {М.},
  publisher = {ИПК Издательство стандартов},
  series = {Единая система программной документации},
  year = {2001}
}

@book{gost505,
  title = {ГОСТ 19.505-79 Руководство оператора. Требования к содержанию и оформлению},
  address = {М.},
  publisher = {ИПК Издательство стандартов},
  series = {Единая система программной документации},
  year = {2001}
}

@book{gost603,
  title = {ГОСТ 19.603-78 Общие правила внесения изменений},
  address = {М.},
  publisher = {ИПК Издательство стандартов},
  series = {Единая система программной документации},
  year = {2001}
}

@book{gost604,
  title = {ГОСТ 19.604-78 Правила внесения изменений в программные документы, выполненные печатным способом},
  address = {М.},
  publisher = {ИПК Издательство стандартов},
  series = {Единая система программной документации},
  year = {2001}
}

\end{filecontents}

\begin{filecontents}{uspd.sty}
\usepackage{rotating}
\usepackage{ifthen}
\hoffset=-5.4mm
\voffset=-20.4mm
\marginparwidth=0mm
\marginparsep=0mm
\marginparpush=0mm
\evensidemargin=0mm
\oddsidemargin=0mm
\topmargin=0mm
\headheight=15mm
\headsep=5mm
\textwidth=180mm
\textheight=247mm
\footskip=10mm
\setlength{\unitlength}{1mm}

\def\ps@uspd{
  \def\@oddfoot{\vbox{\editboxm{\rightmark}}}
  \def\@oddhead{
    \vbox{
    \vbox{\hfil\textnormal{\thepage}\hfil}
    {\vskip 5mm}
    \vbox{\hfil\textnormal{\rightmark~01-1}\hfil}
  }
 }
}

\let\stdsection\section
\renewcommand\section{\newpage\stdsection}

\newcommand\editboxm[1]{
\begin{tabular}{|c|c|c|c|c|}
\hline
 & & & & \\
\hline
Изм. & Лист & № докум. & Подп. & Дата \\
\hline
 #1 & & & & \\
\hline
Инв. № подл. & Подп. и дата & Взам. инв. № & Инв. № дубл. & Подп. и дата \\
\hline
\end{tabular}
}

\newcommand\editbox[1]{{
\newpage
\noindent\begin{picture}(0,0)(0,0)
\put(-12,-252){\framebox(12,145)}
\put(-7,-252){\line(0,1){145}}
\put(-12,-142){\line(1,0){12}}
\put(-12,-167){\line(1,0){12}}
\put(-12,-192){\line(1,0){12}}
\put(-12,-227){\line(1,0){12}}
\put(-11,-251){\begin{sideways}\parbox[t]{23mm}{\small Инв. {№} подп.}\end{sideways}}
\put(-11,-226){\begin{sideways}\parbox[t]{33mm}{\small Подп. и дата}\end{sideways}}
\put(-11,-191){\begin{sideways}\parbox[t]{23mm}{\small Взам инв. {№}}\end{sideways}}
\put(-11,-166){\begin{sideways}\parbox[t]{23mm}{\small Инв. {№} дубл.}\end{sideways}}
\put(-11,-141){\begin{sideways}\parbox[t]{33mm}{\small Подп. и дата}\end{sideways}}
\end{picture}
#1
\newpage
}}

\newcommand\pagesfield[1]{\ifthenelse{\pageref{#1} > 1}{Листов \pageref{#1}}{}}

\def\vhrulefill#1{\makebox[#1][l]{\leavevmode\leaders\hrule\@height0.5pt\hfill \kern\z@}}

\newcommand{\signature}[3]{%
\parbox{0.4\textwidth}{%
\begin{center}
#1
\par
#2
\par
\vhrulefill{7em}~#3
\par
«\vhrulefill{2em}»~\vhrulefill{5em}~{\number\year}~г.
\end{center}
}}

\newcommand\UspdApprovalPage[1]{\thispagestyle{empty}\editbox{

\begin{center}
\textbf{\@organization} \\
\@organizationunit \\


\@agreed
\@approved

\vskip 20mm

\makeatletter
\@project
\vskip 1em
\@title
\makeatother

\vskip 7mm

{\large ЛИСТ УТВЕРЖДЕНИЯ}

\vskip 10mm

{{#1} 01-1-ЛУ}

\pagesfield{endlu}

\vskip 10mm

\begin{flushright}
\@performer
\end{flushright}

\vfill
{{\number\year}}
\end{center}
\label{endlu}
}

\newcommand\UspdTitlePage[1]{{\setcounter{page}{1}\thispagestyle{empty}\editbox{

{УТВЕРЖДЕНО}

{{#1} 01-1-ЛУ}

\vskip 10mm

\vskip 20mm

\begin{center}

\@project
\vskip 1em
\@title
\end{center}

\vskip 20mm

\begin{center}
{{#1}}
\end{center}
\begin{center}
\pagesfield{enddoc}
\end{center}
\vfill
\begin{center}
{{\number\year}}
\end{center}
}}
}
\renewcommand{\rightmark}{\normalfont{#1}}
\pagestyle{uspd}
\newpage
}

\newcommand\UspdChangesList{{
\newpage
\thispagestyle{empty}
\noindent\begin{picture}(185,247)(0,0)
\put(0,0){\line(0,1){247}}
\put(8,0){\line(0,1){232}}
\put(28,0){\line(0,1){232}}
\put(48,0){\line(0,1){232}}
\put(68,0){\line(0,1){232}}
\put(88,0){\line(0,1){237}}
\put(108,0){\line(0,1){237}}
\put(133,0){\line(0,1){237}}
\put(158,0){\line(0,1){237}}
\put(173,0){\line(0,1){237}}
\put(185,0){\line(0,1){247}}
\put(0,247){\line(1,0){185}}
\put(0,237){\line(1,0){185}}
\put(0,232){\line(1,0){88}}
\put(0,212){\line(1,0){185}}
\put(0,0){\line(1,0){185}}
\multiput(0,207)(0,-5){41}{\line(1,0){185}}
\put(0,237){\makebox[185mm][c]{\raisebox{2\depth}{\Large {Лист регистрации изменений}}}}
\put(0,232){\makebox[88mm][c]{\raisebox{1.5\depth}{\small Номера листов (страниц)}}}
\put(0.5,227){\parbox[t]{7mm}{\small Изм}}
\put(9,227){\parbox[t]{18mm}{\small из\-ме\-нён\-ных}}
\put(29,227){\parbox[t]{18mm}{\small за\-ме\-нён\-ных}}
\put(49,227){\parbox[t]{18mm}{\small но\-вых}}
\put(69,227){\parbox[t]{18mm}{\small ан\-ну\-ли\-ро\-ван\-ных}}
\put(89,232){\parbox[t]{18mm}{\small Всего листов (страниц) в докум}}
\put(109,232){\parbox[t]{23mm}{\small {№} документа}}
\put(134,232){\parbox[t]{23mm}{\small Входящий {№} сопроводительного докум и дата}}
\put(159,232){\parbox[t]{13mm}{\small Подп}}
\put(174,232){\parbox[t]{10mm}{\small Дата}}
\label{enddoc}
\addcontentsline{toc}{section}{Лист регистрации изменений}
\end{picture}
\newpage
}}

\makeatletter
\newcommand\agreed[2]{\def\@agreed{\signature{СОГЛАСОВАНО}{#1}{#2}}}
\newcommand\approved[2]{\def\@approved{\signature{УТВЕРЖДАЮ}{#1}{#2}}}
\newcommand\performer[2]{\def\@performer{\signature{Исполнитель}{#1}{#2}}}

\newcommand\abstracts[1]{\def\@abstracts{#1}}
\newcommand{\project}[1]{\def\@project{#1}}
\newcommand{\organization}[1]{\def\@organization{#1}}
\newcommand{\organizationunit}[1]{\def\@organizationunit{#1}}
\newenvironment{uspd}[1]{\sloppy\UspdApprovalPage{#1}\UspdTitlePage{#1}
\section*{Аннотация}{\@abstracts}
\newpage\tableofcontents\newpage}{}
\makeatother
\end{filecontents}

\documentclass[a4paper,12pt]{article}
\usepackage{ucs}
\usepackage{fontspec}
\usepackage{polyglossia}
\usepackage{hyperref}
\usepackage{amsmath, amsthm, amssymb}
\usepackage{setspace}
\usepackage[normalem]{ulem}
\usepackage{listings}
\usepackage{multirow}
\usepackage{float}
\usepackage[section, numbib, numindex]{tocbibind}
\usepackage{uspd}

\bibliographystyle{ugost2008}

\setdefaultlanguage{russian}
\setotherlanguages{english}

\setmainfont{Times}
\newfontfamily\cyrillicfont{Times}
\setmonofont{DejaVu Sans Mono}

\lstset{inputencoding=utf8,
 basicstyle=\footnotesize,
 extendedchars=\true}

\begin{document}

\sloppy


\project{Компонент расширения приложения Redmine для автоматической генерации документов по ГОСТ ЕСПД}
\organization{Национальный исследовательский университет

«Высшая школа экономики»}
\organizationunit{Факультет компьютерных наук

Департамент программной инженении}

\agreed{1}{shersh}
\approved{2}{shil}
\performer{Student}{Victor}

\title{Техническое задание}
\abstracts{Техническое задание – это основной документ, оговаривающий набор требований и порядок создания программного продукта, в соответствии с которым производится разработка программы, ее тестирование и приемка.

Документ Техническое задание содержит следующие разделы:
«Введение», «Основания для разработки», «Назначение разработки», «Требования к программе или программному изделию», «Требования к программной документации», «Технико-экономические показатели», «Стадии и этапы разработки», «Порядок контроля и приёмки».

В разделе «Введение» указано наименование и краткая характеристика области применения \{\{name1\}\}.

В разделе «Основания для разработки» указан документ на основании, которого ведется разработка и наименование темы разработки.

В разделе «Назначение разработки» указано функциональное и эксплуатационное назначение программного продукта.

Раздел «Требования к программе» содержит основные требования к функциональным характеристикам, к надежности, к условиям эксплуатации, к составу и параметрам технических средств, к информационной и программной совместимости, к маркировке и упаковке, к транспортировке и хранению, а также специальные требования.









Настоящий документ разработан в соответствии с требованиями:

\begin{enumerate}
  \item ГОСТ 19.101-77 Виды программ и программных документов \cite{gost101}
  \item ГОСТ 19.102-77 Стадии разработки \cite{gost102}
  \item ГОСТ 19.103-77 Обозначения программ и программных документов \cite{gost103}
  \item ГОСТ 19.104-78 Основные надписи \cite{gost104}
  \item ГОСТ 19.105-78 Общие требования к программным документам \cite{gost105}
  \item ГОСТ 19.106-78 Требования к программным документам, выполненным печатным способом \cite{gost106}
  \item ГОСТ 19.201-78 Техническое задание. Требования к содержанию и оформлению \cite{gost201}
\end{enumerate}

Изменения к данному документу оформляются согласно ГОСТ 19.603-78 \cite{gost603}, ГОСТ 19.604-78 \cite{gost604}.}

\begin{uspd}{RU.17701729.502290-01 ТЗ}
\section{Введение}
malta
    \subsection{Название программы}

    
\subsection{Краткая характеристика области применения}

    

\section{Основания для разработки}

    \subsection{Документы, на основании которых ведется разработка}

    
\subsection{Наименование темы разработки}

    

\section{Назначение разработки}

    \subsection{Функциональное назначение}

    
\subsection{Экспулатационное назначение}

    

\section{Требования к программе или программному изделию}

    \subsection{Требования к функциональным характеристикам}

    \subsubsection{Требования к составу выполняемых функций}

    
\subsubsection{Требования к организации входных данных}

    
\subsubsection{Требования к организации выходных данных}

    
\subsubsection{Требования к временным характеристикам}

    

\subsection{Требования к интерфейсу}

    
\subsection{Требования к надёжности}

    \subsubsection{Требования к обеспечению надежного (устойчивого) функционирования программы}

    
\subsubsection{Время восстановления после отказа}

    
\subsubsection{Отказы из-за некорректных действий оператора}

    

\subsection{Условия эксплуатации}

    \subsubsection{Климатические условия эксплуатации}

    
\subsubsection{Требования к видам обслуживания}

    
\subsubsection{Требования к численности и квалификации персонала}

    
\subsubsection{Требования к составу и параметрам технических средств}

    

\subsection{Требования к информационной и программной совместимости}

    \subsubsection{Требования к информационным структурам и методам решения}

    
\subsubsection{Требования к программным средствам, используемым программой}

    
\subsubsection{Требования к исходным кодам и языкам программирования}

    
\subsubsection{Требования к защите информации и программы}

    

\subsection{Требования к маркировке и упаковке}

    
\subsection{Требования к транспортировке и хранению}

    \subsubsection{Требования к хранению и транспортировке компакт-дисков (CD)}

    
\subsubsection{Требования к хранению и транспортировке программных документов, предоставляемых в печатном виде}

    

\subsection{Специальные требования}

    

\section{Требования к программной документации}

    \subsection{Предварительный состав программной документации}

    
\subsection{Специальные требования к программной документации}

    

\section{Технико-экономические показатели}

    \subsection{Ориентировочная экономическая эффективность}

    
\subsection{Предполагаемая потребность}

    
\subsection{Экономические преимущества разработки по сравнению с отечественными и зарубежными образцами или аналогами}

    

\section{Стадии и этапы разработки}

    
\section{Порядок контроля и приёмки}

    \subsection{Виды испытаний}

    
\subsection{Общие требования к приёмке работы}

    



\bibliography{uspd}
\end{uspd}
\end{document}
